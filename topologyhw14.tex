\documentclass[10pt]{article}
\setlength\headheight{14.5pt}
\title{Homework}
\author{Frederick Robinson}
\date{17 May 2010}


\input xy
\xyoption{all}

\usepackage{amsfonts}
\usepackage{fancyhdr}
\usepackage{amsthm}
\pagestyle{fancyplain}

\newtheorem{theorem}{Theorem}
\newtheorem{lemma}[theorem]{Lemma}

\begin{document}

\lhead{Frederick Robinson}
\rhead{Math 344: Topology}

   \maketitle

\section{Chapter 8 Section 3}

\subsection{Problem 11}
\subsubsection{Question}
Calculate the homology groups of the following complexes: (a) three copies of the boundary of a triangle all joined together at a vertex; (b) two hollow tetrahedra glued together along an edge; (c) a complex whose polyhedron is homeomorphic to the M\"obius strip; (d) a complex which triangulates the cyclinder.
\subsubsection{Answer}


\subsection{Problem 16}
\subsubsection{Question}
Calculate the homology group of a triangulation of the sphere with $k$ holes.
\subsubsection{Answer}

\section{Chapter 8 Section 5}

\subsection{Problem 20}
\subsubsection{Question}
Prove lemma (8.5)

If $\psi: C(L) \to C(M)$ is a second chain map then $\psi \circ \phi: C(K) \to C(M) $ is a chain map and $(\psi \circ \phi)_* = \psi_* \circ \phi_*: H_q(K) \to H_q(M)$.
\subsubsection{Answer}
The following diagrams commute for all $q$.
\[
\xymatrix{C_q(K ) \ar[r]^{\phi_q} \ar[d]_\partial & C_q(L)  \ar[d]_\partial \\
C_{q-1}(K) \ar[r]^{\phi_{q-1}}& C_{q-1}(L)}
\quad
\xymatrix{C_q(L ) \ar[r]^{\psi_q} \ar[d]_\partial & C_q(M)  \ar[d]_\partial \\
C_{q-1}(L) \ar[r]^{\psi_{q-1}}& C_{q-1}(M)}
\]
Therefore, the following diagram commutes, again for all $q$.
\[
\xymatrix{C_q(K ) \ar[r]^{\phi_q} \ar[d]_\partial  & C_q(L ) \ar[r]^{\psi_q} \ar[d]_\partial & C_q(M)  \ar[d]_\partial \\
C_{q-1}(K) \ar[r]^{\phi_{q-1}}& C_{q-1}(L) \ar[r]^{\psi_{q-1}}& C_{q-1}(M)
}\]
Rewriting we have what we wanted to prove
\[
\xymatrix{C_q(K ) \ar[r]^{\psi_q \circ \phi_q} \ar[d]_\partial  & C_q(M)  \ar[d]_\partial \\
C_{q-1}(K) \ar[r]^{\psi_{q-1} \circ \phi_{q-1}}& C_{q-1}(M)
}\]

Now 
\begin{eqnarray*}
(\psi \circ \phi)_* H_q(K)&=&(\psi \circ \phi)_*  \ker{\partial} C_q(K) / \mathrm{Im\ }{\partial C_{q+1}(K) }\\
&=&  \ker{\partial} (\psi \circ \phi)C_q(K) / \mathrm{Im\ }{\partial (\psi \circ \phi) C_{q+1}(K) }\\
&=&  \ker{\partial} \psi C_q(K) / \mathrm{Im\ }{\partial \psi C_{q+1}(K) } \circ  \ker{\partial} \phi C_q(K) / \mathrm{Im\ }{\partial  \phi C_{q+1}(K) } \\
&=&  (\psi _* \circ \phi _*) H_q(K)
\end{eqnarray*}

\subsection{Problem 21}
\subsubsection{Question}
Check that the subdivision map $\chi: C(K) \to C(K') $ is a chain map.
\subsubsection{Answer}
Let $\sigma$ be a typical oriented $q$-simplex of $K$. In particular say $\sigma = (v_0, v_1, \dots, v_k, v_{k+1}, \dots, v_q)$. Now we compute
\[ \chi_q(\sigma) = \sum_{i=0}^k (-1)^i (v,v_0, \dots, \hat{v_i}, \dots, v_k, v_k+1, \dots, v_q)\]
and
\[\partial \chi_q(\sigma) = \sum_{j=0}^q \sum_{i=0}^k (-1)^i (v,v_0, \dots,\hat{v_j},\dots, \hat{v_i}, \dots, v_k, v_k+1, \dots, v_q) .\]
Finally we compute
\[\partial (\sigma)= \sum_{j=0}^q  (v_0, \dots,\hat{v_j}, \dots, v_k, v_k+1, \dots, v_q) \]
and
\[\partial \chi_q(\sigma) =  \sum_{i=0}^k \sum_{j=0}^q (-1)^i (v,v_0, \dots,\hat{v_j},\dots, \hat{v_i}, \dots, v_k, v_k+1, \dots, v_q) .\]
So interchanging the order of summation we see that $\partial \chi_q = \chi_{q-1} \partial$ as desired.
\end{document}
