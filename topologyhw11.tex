\documentclass[10pt]{article}
\setlength\headheight{14.5pt}
\title{Homework}
\author{Frederick Robinson}
\date{21 April 2010}
\usepackage{amsfonts}
\usepackage{fancyhdr}
\usepackage{amsthm}
\pagestyle{fancyplain}

\newtheorem{theorem}{Theorem}
\newtheorem{lemma}[theorem]{Lemma}

\begin{document}

\lhead{Frederick Robinson}
\rhead{Math 344: Topology}

   \maketitle

\section{Chapter 5 Section 4}

\subsection{Problem 25}
\subsubsection{Question}
Show that the punctured torus deformation-retracts onto the one-point union of two circles.
\subsubsection{Answer}
\begin{proof}This is clear.\end{proof}
\section{Chapter 6 Section 4}

\subsection{Problem 20}
\subsubsection{Question}
Use van Kampen's theorem to calculate the fundamental group of the double torus by dividing the surface into two halves, each of which is a punctured torus. Do the calculation again, this time splitting the surface into a disc and the closure of the complement of the disc
\subsubsection{Answer}
\begin{enumerate}
\item The fundamental group of a punctured torus is $\mathbb{Z} * \mathbb{Z}$. So, the fundamental group of a double torus is $\mathbb{Z} * \mathbb{Z}*\mathbb{Z} * \mathbb{Z}$ modulo elements of the form $b c^{-1}$ where $b$ is the generator for one element of the free product and $c$ generates the other. However $b c^{-1}= e \Rightarrow b=c $ so the fundamental group is just $\mathbb{Z} * \mathbb{Z}* \mathbb{Z}$
\item The double torus minus a disc deformation retracts onto a chain of three circles linked each to the next. This has fundamental group $\mathbb{Z} * \mathbb{Z}* \mathbb{Z}$. The disc is simply connected though, so the fundamental group of the double torus is just $\mathbb{Z} * \mathbb{Z} * \mathbb{Z}$.
\end{enumerate}


\subsection{Problem 22}
\subsubsection{Question}
Prove that the `dunce hat' (Fig. 5.11) is simply connected using van Kampen's theorem
\subsubsection{Answer}
\begin{proof}
If we think divide the dunce hat into its interior and its edges we see that the interior has trivial fundamental group while the edges have fundamental group $\mathbb{Z}$. However, the loop which generates the fundamental group the intersection of our two pieces also generates the fundamental group of the edge. Hence, the fundamental group of the entire dunce hat is just the trivial group, and it is simply connected as desired.
\end{proof}

\subsection{Problem 23}
\subsubsection{Question}
Let $X$ be a path-connected triangulable space. How does attaching a disc to $X$ affect the fundamental group of $X$?
\subsubsection{Answer}
By Seifert-van Kampen the fundamental group is $\pi_1(X) / N$ where $N$ is as usual. In particular $N$ is just generated by loops in $X$ which reside within the border where the disc is attached
\end{document}
