\documentclass[10pt]{article}
\setlength\headheight{14.5pt}
\title{Homework}
\author{Frederick Robinson}
\date{8 March 2010}
\usepackage{amsfonts}
\usepackage{fancyhdr}
\usepackage{amsthm}
\pagestyle{fancyplain}

\newtheorem{theorem}{Theorem}
\newtheorem{lemma}[theorem]{Lemma}

\begin{document}

\lhead{Frederick Robinson}
\rhead{Math 344: Topology}

   \maketitle



\section{Chapter 7 Section 4}

\subsection{Problem 17}

\subsubsection{Question}
The straight lines shown in Fig. 7.16 represent three simple closed curves in the Klein bottle. Thicken each curve, decide whether the result is a cylinder or a M\"{o}bius strip, and describe the effect of doing surgery along the curve.
\subsubsection{Answer}
Let's call the diagonal cut 1, the cut which connects the sides which are oriented in the same direction 2, and the cut which connects sides oriented in opposite directions 3.

\begin{enumerate}
\item If we thicken the first line we get a M\"{o}bius strip. Thus, doing surgery along this line results in the projective plane.
\item If we thicken this line we get a cylinder. Doing surgery along this line therefore results in sphere.
\item Thickening this line results in a M\"{o}bius strip. We may therefore conclude again that  doing surgery along this line results in the projective plane.
\end{enumerate}

\subsection{Problem 18}

\subsubsection{Question}
Show that the surface illustrated in Fig. 7.17 is homeomorphic to one of the standard ones using the procedure of theorem (7.14).
\subsubsection{Answer}
The surface in question is homeomorphic to a sphere with 5 handles. See attached illustration. 
\subsection{Problem 19}

\subsubsection{Question}
Let $X \supset Y \supset Z$ be three concentric discs in the plane. Find a homeomorphism from $X$ to itself which is the identity on the boundary circle of $X$ and which throws $Y$ onto $Z$.
\subsubsection{Answer}
Let the discs $X$, $Y$, $Z$ in $\mathbb{R}^2$ be defined in polar coordinates with radii $r_x> r_y>r_z$ respectively. Define a mapping $\varphi: \mathbb{R}^2 \to \mathbb{R}^2$ as below
\[\varphi(r,\theta) = \left\{ \begin{array}{ll}
\displaystyle r \frac{r_z}{r_y }\hat{r} + \theta \hat{\theta} &r < r_y\\ \\
\displaystyle \left( \frac{r r_x-r_x r_y-r r_z+r_x r_z}{r_x-r_y} \right) \hat{r} + \theta \hat{\theta} & r_y<r< r_z\\
\end{array} \right.\]

We check that these functions agree on their border first
\[ \varphi(r=0)=0 \hat{r} \]
\[r_y \frac{r_z}{r_y} = r_z = \left( \frac{r_y r_x-r_x r_y-r_y r_z+r_x r_z}{r_x-r_y} \right) \]
\[ \varphi(r=r_x) = r_x \]

This is a composition of affine linear functions which agree on their boundary and is therefore continuous with continuous inverse. Moreover it is a bijection on the set with $r < r_x$. Therefore it is a homeomorphism.

Finally, it is easy to see in the above computation that it takes each point in with $r<r_y$ onto $r< r_z$ as desired. 





\end{document}
