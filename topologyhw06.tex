\documentclass[10pt]{article}
\setlength\headheight{14.5pt}
\title{Homework}
\author{Frederick Robinson}
\date{21 February 2010}
\usepackage{amsfonts}
\usepackage{fancyhdr}
\usepackage{amsthm}
\pagestyle{fancyplain}

\newtheorem{theorem}{Theorem}
\newtheorem{lemma}[theorem]{Lemma}

\begin{document}

\lhead{Frederick Robinson}
\rhead{Math 344: Topology}

   \maketitle



\section{Chapter 6 Section 1}

\subsection{Problem 1}

\subsubsection{Question}
Construct triangulations for the cylinder, the Klein bottle, and the double torus.
\subsubsection{Answer}
See attached drawings.

\subsection{Problem 3}

\subsubsection{Question}
If $|K|$ is a connected space, show that any two vertices of $K$ can be connected by a path whose image is a collection of vertices and edges of $K$.
\subsubsection{Answer}
We can divide the vertices of a simplicial complex $|K|$ into equivalence classes $X_i$ where a member of an equivalence class is connected by edges to each other member.

\begin{proof}
Let $r, s, t$ be vertices of $K$.

\emph{Symmetric}
If $r$ is connected to  $s$ by an edge then $s$ is connected to $r$

\emph{Reflexive }
$s$ is related to itself by definition

\emph{Transitive}
if $r$ is connected to $s$ and $s$ is connected to $t$ then $r$ is connected to $t$ by edges  (in particular the union of the edge(s) connecting $s$ to $r$ and the edge(s) connecting $r$ to $t$)

\end{proof}

I claim that there is only one such class $X_1$.

\begin{proof}
Each vertex of the same simplicial complex is in the same equivalence class. Take an vertices $r$ and  say it has barycentric coordinates $(0,0,\dots,1,\dots,0)$. We can take a path 
\[ \varphi(t) =(0,0,\dots,1-t,\dots,0) \quad t\in [0,1]\]
along an edge to the vertex having barycentric coordinates $(0,0,\dots,0)$. So, since each vertex is in the same equivalence class as one fixed vertex they are all in the same class by definition of equivalence class.

Now I claim that given two simplexes $T_i$ and $T_j$ in the same (connected) simplicial complex their vertices are members of the same equivalence class. 

Say that our simplicial complex consists of the intersection of each of $T_i$ simplexes. Since it is connected we know that any $T_i$ has nontrivial intersection with some other $T_j$ and moreover that there is only one equivalence class of simplexes where we define two simplexes to be in the same equivalence class if they have nontrivial intersection (and transitivity). 

Also, note that if two simplexes $T_i$ and $T_j$ have nontrivial intersection their vertices are in the same equivalence class of vertices since, they intersect on a face and therefore they in particular \emph{share} at least one vertex.

So, these facts imply that each vertex is in the same equivalence class and, any two vertices can be connected by a path whose image is a collection of edges and vertices only as claimed.
\end{proof}

\subsection{Problem 7}

\subsubsection{Question}
Show that $S^n$ and $P^n$ are both triangulable.
\subsubsection{Answer}
$S^n$ and $P^n$ are both Riemann surfaces. Thus, they are therefore triangulable. To quote Wikipedia:

''There is a standard proof that smooth closed surfaces can be triangulated (see Jost 1997). Indeed, if the surface is given a Riemannian metric, each point x is contained inside a small convex geodesic triangle lying inside a normal ball with centre x. The interiors of finitely many of the triangles will cover the surface; since edges of different triangles either coincide or intersect transversally, this finite set of triangles can be used iteratively to construct a triangulation."

I have attached the proof from Jost.

I will demonstrate this in particular for $S^n$.

\begin{proof}
\[ (B^n)^o \cong S^{n-1}\]
\[K^n \cong B^n \Rightarrow (K^n)^o \cong ( B^n)^o \cong S^{n-1}\]
Where $K^n$ is the $n$-simplex. So, since boundary of a simplex is a simplicial complex we have demonstrated what we wanted to show.
\end{proof}
\end{document}
